\documentclass[12pt]{book} 

\usepackage{geometry}
\geometry{
	paper=a4paper, 
	top=20mm, 
	bottom=20mm,  
	right=25mm, 
	nohead, 
	nofoot,  
	%showframe, 
}

%\usepackage{lmodern}
%\usepackage{anyfontsize}
%\usepackage{t1enc}
%\usepackage{mathptmx}
\usepackage[T1,T2A]{fontenc}
\usepackage[utf8]{inputenc}
\usepackage[unicode, pdftex]{hyperref}
\usepackage[russian, english]{babel}
\usepackage{amsmath,amsfonts,amssymb,amsthm} 
\usepackage{setspace}
\usepackage{mathtools}
	
%\usepackage{fontspec}
%\setmainfont{cmun}[
%Extension=.otf,
%UprightFont=*rm,
%ItalicFont=*ti,
%BoldFont=*bx,
%BoldItalicFont=*bi]


\newtheorem*{task}{Задача}

\newenvironment{solution}{\paragraph{Решение:}}{\hfill$\square$} 

\begin{document}
\fontsize{12}{12}\selectfont

\title{\bf \huge Часть I}
\date{Update on \today}
\maketitle 

\begin{task}
В ЕГЭ принимают участие $25$ школьников. Экзамен состоит из нескольких вопросов, на каждый из которых можно дать один из пяти вариантов ответа. Оказалось, что любые два школьника не более чем на один вопрос ответили одинаково. Докажите, что в ЕГЭ было не больше $6$ вопросов. (почти решено)
\end{task}

\begin{task}
В правильном $(6n + 1)$ -угольнике $k$ вершин покрашены в красный цвет, а остальные - в синий. Докажите, что количество равнобедренных треугольников с одноцветными вершинами не зависит от способа раскраски. (Подсчет двумя способами - треугольников и ребер каждой раскраски)
\end{task}

\begin{task}
В сенате, состоящем из $30$ сенаторов, каждые двое дружат или враждуют, причём каждый враждует ровно с $6$ другими. Найдите общее количество троек сенаторов, в которых либо все трое дружат друг с другом, либо все трое враждуют между собой. (Подсчет двумя способами. Подсчет треугольников)
\end{task}

\begin{task}
Имеется $1990$ кучек, состоящих соответственно из $1, 2, 3, \dots, 1990$ камней. За один шаг разрешено выбросить из любого множества кучек по одинаковому числу камней. За какое наименьшее число шагов можно выбросить все камни? 
\end{task}

\begin{task}
Найдите наибольшее число цветов, в которые можно покрасить рёбра куба (каждое ребро одним цветом) так, чтобы для каждой пары цветов нашлись два соседних ребра, покрашенные в эти цвета. Соседними считаются рёбра, имеющие общую вершину.
\end{task}

\begin{task}
Король вызвал двух мудрецов и объявил им задание: первый задумывает семь различных натуральных чисел с суммой $100$, тайно сообщает их королю, а второму мудрецу называет лишь четвёртое по величине из этих чисел, после чего второй должен отгадать задуманные числа. У мудрецов нет возможности сговориться. Могут ли мудрецы гарантированно справиться с заданием? 
\end{task}

\begin{task}
Фокусник с завязанными глазами выдаёт зрителю $29$ карточек с номерами от $1$ до $29$. Зритель прячет две карточки, а остальные отдаёт ассистенту фокусника. Ассистент указывает зрителю на две из них, и зритель называет номера этих карточек фокуснику (в том порядке, в каком захочет). После этого фокусник угадывает номера карточек, спрятанных у зрителя. Как фокуснику и ассистенту договориться, чтобы фокус всегда удавался? (два способа)
\end{task}

\begin{task}
Дракон запер в пещере шестерых гномов и сказал: "У меня есть семь колпаков семи цветов радуги. Завтра утром я завяжу вам глаза и надену на каждого по колпаку, а один колпак спрячу. Затем сниму повязки, и вы сможете увидеть колпаки на головах у других, но общаться я вам уже не позволю. После этого каждый втайне от других скажет мне цвет спрятанного колпака. Если угадают хотя бы трое, всех отпущу. Если меньше - съем на обед". Как гномам заранее договориться действовать, чтобы спастись?
\end{task}

\begin{task}
Султан собрал $300$ придворных мудрецов и предложил им испытание. Он сообщил им список из $25$ цветов и сказал, что на испытании каждому мудрецу наденут на голову колпак одного из этих цветов, причем если для каждого цвета написать количество надетых колпаков этого цвета, то все числа будут различны. Каждый мудрец увидит, какой колпак на ком надет, но свой колпак не увидит. Затем одновременно (по сигналу) каждый должен будет назвать предполагаемый цвет своего колпака. Могут ли мудрецы заранее договориться действовать так, чтобы гарантированно хотя бы $150$ из них назвали цвет верно?
\end{task}

\begin{task}
Одиннадцати мудрецам завязывают глаза и надевают каждому на голову колпак одного из $1000$ цветов. После этого им глаза развязывают, и каждый видит все колпаки, кроме своего. Затем одновременно каждый показывает остальным одну из двух карточек - белую или чёрную. После этого все должны одновременно назвать цвет своих колпаков. Удастся ли это? Мудрецы могут заранее договориться о своих действиях (до того, как им завязали глаза); мудрецам известно, каких $1000$ цветов могут быть колпаки. 
\end{task}

\end{document}

\documentclass[12pt]{article} 

\usepackage{geometry}
\geometry{
	paper=a4paper, 
	top=2cm, 
	bottom=2cm, 
	left=2cm, 
	right=2cm, 
	headheight=14pt, 
	footskip=1.4cm, 
	headsep=10pt, 
	%showframe, 
}

\usepackage{lmodern}
\usepackage{anyfontsize}
\usepackage{t1enc}
\usepackage{mathptmx}
\usepackage[T1,T2A]{fontenc}
\usepackage[utf8]{inputenc}
\usepackage[russian,english]{babel}
\usepackage{amsmath,amsfonts,amssymb,amsthm} 
\usepackage{mathtools}
\usepackage{fontspec}
\setmainfont{cmun}[
Extension=.otf,
UprightFont=*rm,
ItalicFont=*ti,
BoldFont=*bx,
BoldItalicFont=*bi] 

\begin{document}
\fontsize{12}{12}\selectfont

\title{\bf \huge Часть I}
\date{Update on \today}
\maketitle 



\section*{Теория графов}

\begin{task}
Докажите, что среди любых шести человек есть либо трое попарно знакомых, либо трое попарно незнакомых.
\end{task}

\noindent\textbf{Решение.} 

\begin{task}
Каждое из рёбер полного графа с $17$ вершинами покрашено в один из трёх цветов. Докажите, что есть три вершины, все рёбра между которыми - одного цвета.
\end{task}

\begin{task}
У любого выпуклого многогранника есть две грани с одинаковым числом сторон. Докажите это.
\end{task}

\begin{task}
Докажите, что для плоского связного графа справедливо неравенство  $E \leqslant  3V - 6$.
\end{task}

\begin{task}
Докажите, что граф, имеющий $10$ вершин, степень каждой из которых равна $5$, - не плоский.    
\end{task}

\begin{task}
Докажите, что граф, имеющий пять вершин, каждая из которых соединена ребром со всеми остальными, не является плоским.
\end{task}

\begin{task}
Докажите, что в плоском графе есть вершина, степень которой не превосходит $5$. 
\end{task}

\begin{task}
Каждое ребро полного графа с $11$ вершинами покрашено в один из двух цветов: красный или синий. Докажите, что либо красный, либо синий граф не является плоским.
\end{task}

\begin{task}
В квадрате отметили $20$ точек и соединили их непересекающимися отрезками друг с другом и с вершинами квадрата так, что квадрат разбился на треугольники. Сколько получилось треугольников? 
\end{task}

\section*{Числовые таблицы и их свойства}


\begin{task} В таблицу $9 \times 9$ вписаны все целые числа от $1$ до $81$. Доказать, что найдутся два соседних числа, разность между которыми не меньше $6$. 
\end{task}

\begin{task} 
Вася записал в клетки таблицы $9 \times 9$ натуральные числа от $1$ до $81$ (в каждой клетке стоит по числу, все числа различны). Оказалось, что любые два числа, отличающиеся на $3$, стоят в соседних по стороне клетках. Верно ли, что обязательно найдутся две угловых клетки, разность чисел в которых делится на $6$?
\end{task}




\section*{Принцип Дирихле}

\begin{task}
В банде $50$ бандитов. Все вместе они ни в одной разборке ни разу не участвовали, а каждые двое встречались на разборках ровно по разу. Докажите, что один из бандитов был не менее, чем на восьми разборках. 
\end{task}

\begin{task} Каждый из учеников класса занимается не более чем в двух кружках, причём для любой пары учеников существует кружок, в котором они занимаются вместе. Докажите, что найдётся кружок, в котором занимается не менее $\frac{2}{3}$ всего класса.
\end{task}

\begin{task} Из целых чисел от $1$ до $100$ удалили $k$ чисел. Обязательно ли среди оставшихся чисел можно выбрать $k$ различных чисел с суммой 100, если $k = 9$. А если $k = 8$?
\end{task}

\begin{task} Какое наименьшее количество чисел можно вычеркнуть из последовательности $1, 2, 3, \dots, 1982$, чтобы ни одно из оставшихся чисел не равнялось произведению двух других оставшихся чисел?
\end{task}

\begin{task} Вершины правильного стоугольника покрашены в $10$ цветов. Докажите, что можно выбрать $4$ вершины этого стоугольника, являющиеся вершинами прямоугольника и покрашенные не более чем в два различных цвета.
\end{task}

\begin{task} Числа $1, 2, \dots, 100$ стоят по кругу в некотором порядке. Может ли случиться, что у любых двух соседних чисел модуль разности не меньше $30$, но не больше $50$?
\end{task}

\begin{task}
Среди натуральных чисел от 1 до 1200 выбрали 372 различных числа так, что никакие два из них не различаются на 4, 5 или 9. Докажите, что число 600 является одним из выбранных. 
\end{task}


\section*{Диофантовы уравнения}

\begin{task}
Докажите, что уравнение $xy(x-y)+yz(y-z)+zx(z-x) = 6 $ имеет бесконечно много решений в целых числах.
\end{task}

\begin{task}
Уравнение $(x + y + z)^{2} = x^{2} + y^{2} + z^{2}$ имеет бесконечно много решений в целых числах. Докажите это.
\end{task}

\end{document}
 

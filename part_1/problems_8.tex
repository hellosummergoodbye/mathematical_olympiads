\documentclass[12pt]{article} 

\usepackage{geometry}
\geometry{
	paper=a4paper, 
	top=20mm, 
	bottom=20mm,  
	right=25mm, 
	nohead, 
	nofoot,  
	%showframe, 
}

%\usepackage{lmodern}
%\usepackage{anyfontsize}
%\usepackage{t1enc}
%\usepackage{mathptmx}
\usepackage[T1,T2A]{fontenc}
\usepackage[utf8]{inputenc}
\usepackage[unicode, pdftex]{hyperref}
\usepackage[russian, english]{babel}
\usepackage{amsmath,amsfonts,amssymb,amsthm} 
\usepackage{setspace}
\usepackage{mathtools}
	
%\usepackage{fontspec}
%\setmainfont{cmun}[
%Extension=.otf,
%UprightFont=*rm,
%ItalicFont=*ti,
%BoldFont=*bx,
%BoldItalicFont=*bi]


\newtheorem*{task}{Задача}

\newenvironment{solution}{\paragraph{Решение:}}{\hfill$\square$} 

\begin{document}
\fontsize{12}{12}\selectfont

\title{\bf \huge Часть VIII}
\date{Update on \today}
\maketitle  

\begin{task}
В угловой клетке таблицы $5 \times 5$ стоит плюс, а в остальных клетках стоят минусы. Разрешается в любой строке или любом столбце поменять все знаки на противоположные. Можно ли за несколько таких операций сделать все знаки плюсами?
\end{task}

\begin{task}
Существует ли замкнутая $7$-звенная ломаная, которая пересекает каждое свое звено ровно один раз?
\end{task}

\begin{task}
Могут ли все грани выпуклого многогранника иметь $6$ и более сторон?
\end{task}

\begin{task}
В выпуклом $n$-угольнике никакие три диагонали не пересекаются в одной точке. Сколько точек пересечения у этих диагоналей? (Концы диагоналей не считаются точками пересечения.)
\end{task}

\begin{task}
Двое бросают монетку: один бросил её $10$ раз, другой - $11$. Чему равна вероятность того, что у второго монета упала орлом большее число раз, чем у первого?
\end{task}

\begin{task}
На математической олимпиаде было предложено $20$ задач. На закрытие пришло $20$ школьников. Каждый из них решил по две задачи, причём выяснилось, что среди пришедших каждую задачу решило ровно два школьника. Докажите, что можно так организовать разбор задач, чтобы каждый школьник рассказал одну из решенных им задач, и все задачи были разобраны.
\end{task}

\begin{task}
Докажите, что для любого натурального числа $n$ существует бесконечно много натуральных чисел, не представимых в виде суммы $n$ слагаемых, каждое из которых является $n$-й степенью натурального числа.
\end{task}

\begin{task}
В классе $25$ человек. Известно, что среди любых трёх из них есть двое друзей. Докажите, что есть ученик, у которого не менее $12$ друзей.
\end{task}

\begin{task}
На столе лежат $15$ журналов, полностью покрывая его. Докажите, что можно убрать $7$ журналов так, чтобы оставшиеся покрывали не менее $\frac{8}{15}$ площади стола.
\end{task}

\begin{task}
Можно ли разрезать выпуклый $17$-угольник на $14$ треугольников?
\end{task}

\begin{task}
Из квадрата $128 \times 128$ вырезали одну клетку. Докажите, что эту фигуру можно замостить уголками из трёх клеток.
\end{task}

\begin{task}
Шахматную доску $8 \times 8$ покрыли $32$ прямоугольниками из двух клеток (доминошками). Докажите, что найдутся две доминошки, образующие квадрат $2 \times 2$.
\end{task}

\begin{task}
На окружности отмечено $1000$ точек, каждая окрашена в один из $k$ цветов. Оказалось, что среди любых  пяти попарно перескающихся  отрезков, концами которых являются $10$ различных отмеченных точек, найдутся хотя бы три отрезка, у каждого из которых концы имеют разные цвета. При каком наименьшем $k$ это возможно?
\end{task}

\begin{task}
Какое максимальное число попарно пересекающихся подмножеств можно выбрать в множестве из $100$ элементов?
\end{task}

\begin{task}
В множестве, состоящем из $100$ элементов, выбрали несколько различных трехэлементных подмножеств. Докажите, что если выбрано $101$ подмножество, то среди них найдутся два, имеющие ровно один общий элемент.
\end{task}

\begin{task}
Среди $90$ выпускников одной математической гимназии у каждого не менее $10$ друзей. Докажите, что любой выпускник может пригласить в гости трёх других так, что среди четырёх собравшихся у каждого будет не менее двух друзей.
\end{task}

\begin{task}
Некоторые клетки белого прямоугольника размером $3 \times 7$ произвольным образом покрасили в чёрный цвет. Докажите, что обязательно найдутся четыре клетки одного цвета, центры которых являются вершинами некоторого прямоугольника со сторонами, параллельными сторонам исходного прямоугольника. 
\end{task}

\begin{task}
На острове все страны треугольной формы (границы прямые). Если две страны граничат, то по целой стороне. Докажите, что страны можно раскрасить в $3$ цвета так, что соседние по стороне страны будут покрашены в разные цвета. (индукция, крайний)
\end{task}

\begin{task}
Каждое ли целое число можно представить в виде суммы кубов нескольких целых чисел, среди которых нет одинаковых?
\end{task}

\begin{task}
Петя записал $25$ чисел в клетки квадрата $5 \times 5$. Известно, что их сумма равна $500$. Вася может попросить его назвать сумму чисел в любой клетке и всех ее соседях по
стороне. Может ли Вася за несколько таких вопросов узнать, какое число записано в центральной клетке?
\end{task}

\begin{task}
Существуют ли $10$ различных целых чисел таких, что все суммы, составленные из $9$ из них - точные квадраты?
\end{task}

\begin{task}
Миша стоит в центре круглой лужайке радиуса $100$ метров. Каждую минуту он делает шаг длиной $1$ метр. Перед каждым шагом он объявляет направление, в котором хочет шагнуть. Катя имеет право заставить его сменить направление на противоположное. Может ли Миша действовать так, чтобы в какой-то момент обязательно выйти с лужайки, или Катя всегда сможет ему помешать? 
\end{task}

\begin{task}
Школьники одного класса в сентябре ходили в два туристических похода. В первом походе мальчиков было меньше $\frac{2}{5}$ общего числа участников этого похода, во втором - тоже меньше $\frac{2}{5}$. Каждый из учеников участвовал по крайней мере в одном походе. Докажите, что в этом классе мальчики составляют меньше $\frac{4}{7}$ общего числа учеников.
\end{task}

\end{document}
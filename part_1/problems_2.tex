\documentclass[12pt]{book} 

\usepackage{geometry}
\geometry{
	paper=a4paper, 
	top=2cm, 
	bottom=2cm, 
	left=2cm, 
	right=2cm, 
	headheight=14pt, 
	footskip=1.4cm, 
	headsep=10pt, 
	%showframe, 
}

\usepackage{lmodern}
\usepackage{anyfontsize}
\usepackage{t1enc}
\usepackage{mathptmx}
\usepackage[T1,T2A]{fontenc}
\usepackage[utf8]{inputenc}
\usepackage[russian,english]{babel}
\usepackage{amsmath,amsfonts,amssymb,amsthm} 
\usepackage{mathtools}
\usepackage{fontspec}
\setmainfont{cmun}[
Extension=.otf,
UprightFont=*rm,
ItalicFont=*ti,
BoldFont=*bx,
BoldItalicFont=*bi] 

\begin{document}
\fontsize{12}{12}\selectfont

\title{\bf \huge Часть II}
\date{Update on \today}
\maketitle 


\chapter{Принцип Дирихле}

\begin{task}
Множество $A$ состоит из $n$ различных натуральных чисел, сумма которых равна $n^{2}$. Множество $B$ также состоит из $n$ различных натуральных чисел, сумма которых равна $n^{2}$. Докажите, что найдётся число, которое принадлежит как множеству $A$, так и множеству $B$.
\end{task}

\begin{solution}
Предположим противное. Рассмотрим сумму всех чисел в обоих множествах. Поскольку все числе разные, то она не менее $n \cdot (2n + 1)$. Но по условию она равна $2n^{2}$.
\end{solution}

\begin{task} 
Какое наибольшее количество чисел можно выбрать из набора $1, 2, \dots, 1963$, чтобы сумма никаких двух чисел не делилась на их разность?
\end{task}
 

\section*{Процессы и операции}

\begin{task}
Десяти ребятам положили в тарелки по $100$ макаронин. Есть ребята не хотели и стали играть. Одним действием кто-то из детей перекладывает из своей тарелки по одной макаронине всем другим детям. После какого наименьшего количества действий у всех в тарелках может оказаться разное количество макаронин?
\end{task}

\section*{Числовые таблицы и их свойства}

\begin{task}
В таблицу $10 \times 10$ нужно записать в каком-то порядке цифры  $0, 1, 2, 3, \dots, 9$  так, что каждая цифра встречалась бы $10$ раз. Докажите, что найдётся строка или столбец, в которой (в котором) встречается не меньше четырёх различных чисел. 
\end{task}

\begin{solution}
Рассмотрим следующие величины. $c_{i}$ - количество строк, которые занимает $i$. $r_{i}$ - количество столбцов, которые занимает $i$. 
\end{solution}

\begin{task} Можно ли в таблице $19 \times 19$ отметить несколько клеток так, чтобы во всех квадратах $10 \times 10$ было разное количество отмеченных клеток?
\end{task}

\begin{task} В каждой клетке таблицы $9 \times 9$ записано число, по модулю меньшее $1$. Известно, что сумма чисел в каждом квадратике $2 \times 2$ равна $0$. Докажите, что сумма чисел в таблице меньше $9$. 
\end{task}


\section*{Отсортировать}

\begin{task}
Восемь школьников решали восемь задач. Оказалось, что каждую задачу решили пять школьников. Докажите, что найдутся такие два школьника, что каждую задачу решил хотя бы один из них. 
\end{task}

\begin{solution}

\end{solution}

\begin{task}
Дан клетчатый квадрат $10 \times 10$. Внутри него провели $80$ единичных отрезков по линиям сетки, которые разбили квадрат на $20$ многоугольников равной площади. Докажите, что все эти многоугольники равны.
\end{task}

\begin{task}
Любую дробь $\frac{m}{n}$, где $m$ и $n$ - натуральные числа, $1 < m < n$, можно представить в виде суммы нескольких дробей вида $\frac{1}{q}$, причём таких, что знаменатель каждой следующей из этих дробей делится на знаменатель предыдущей дроби. Докажите это.
\end{task}

\section*{Правильные многоугольники}

\begin{task}
Каково максимальное число попарно непараллельных отрезков с концами в вершинах правильного $n$-угольника? 
\end{task}

\begin{task}
У правильного $1981$-угольника отмечены $64$ вершины. Доказать, что существует трапеция с вершинами в отмеченных точках.
\end{task}

\begin{task}
У правильного $5000$-угольника покрашено $2001$ вершина. Докажите, что найдутся три покрашенные вершины, лежащие в вершинах равнобедренного треугольника. 
\end{task}

\begin{solution}

\end{solution}

\end{document} 

\documentclass[12pt]{article} 

\usepackage{geometry}
\geometry{
	paper=a4paper, 
	top=2cm, 
	bottom=2cm, 
	left=2cm, 
	right=2cm, 
	headheight=14pt, 
	footskip=1.4cm, 
	headsep=10pt, 
	%showframe, 
}

\usepackage{lmodern}
\usepackage{anyfontsize}
\usepackage{t1enc}
\usepackage{mathptmx}
\usepackage[T1,T2A]{fontenc}
\usepackage[utf8]{inputenc}
\usepackage[russian,english]{babel}
\usepackage{amsmath,amsfonts,amssymb,amsthm} 
\usepackage{mathtools}
\usepackage{fontspec}
\setmainfont{cmun}[
Extension=.otf,
UprightFont=*rm,
ItalicFont=*ti,
BoldFont=*bx,
BoldItalicFont=*bi] 

\begin{document}
\fontsize{12}{12}\selectfont

\title{\bf \huge Часть IV}
\date{Update on \today}
\maketitle 

\begin{task}
В каждой клетке доски $8 \times 8$ написали по одному натуральному числу. Оказалось, что при любом разрезании доски на доминошки суммы чисел во всех доминошках будут разные. Может ли оказаться, что наибольшее записанное на доске число не больше $32$?
\end{task}

\begin{task}
У доски $6 \times 6$ вырезали две угловые клетки на диагонали. Можно ли покрыть оставшуюся часть доминошками из двух клеток?
\end{task}

\begin{task}
Незнайка легко замощает доску $10 \times 10$ квадратами $2 \times 2$, а вот полосками из четырёх клеток у него никак не получается. А в принципе это возможно?
\end{task}

\begin{task}
Круг разделен на шесть секторов. В каждом из них лягушка. Каждую минуту какие-то две лягушки перескакивают из своих секторов в соседние. Смогут ли когда-нибудь эти лягушки собраться в одном секторе? 
\end{task}

\begin{task}
Среди любых $n$ попарно взаимно простых чисел, больших $1$ и меньших $(2n - 1)^{2}$, есть хотя бы одно простое число. Докажите это.
\end{task}

\begin{task}
Известно, что число $2^{333}$ имеет $101$ цифру и начинается с цифры $1$. Сколько чисел в ряду $2, 4, 8, 16, \dots, 2^{333}$ начинается с цифры $4$? 
\end{task}

\end{document} 

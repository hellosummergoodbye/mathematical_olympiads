\documentclass[12pt]{article} 

\usepackage{geometry}
\geometry{
	paper=a4paper, 
	top=2cm, 
	bottom=2cm, 
	left=2cm, 
	right=2cm, 
	headheight=14pt, 
	footskip=1.4cm, 
	headsep=10pt, 
	%showframe, 
}

\usepackage{lmodern}
\usepackage{anyfontsize}
\usepackage{t1enc}
\usepackage{mathptmx}
\usepackage[T1,T2A]{fontenc}
\usepackage[utf8]{inputenc}
\usepackage[russian,english]{babel}
\usepackage{amsmath,amsfonts,amssymb,amsthm} 
\usepackage{mathtools}
\usepackage{fontspec}
\setmainfont{cmun}[
Extension=.otf,
UprightFont=*rm,
ItalicFont=*ti,
BoldFont=*bx,
BoldItalicFont=*bi] 

\begin{document}
\fontsize{12}{12}\selectfont

\title{\bf \huge Часть V}
\date{Update on \today}
\maketitle 

\begin{task}
Можно ли из $18$ плиток размером $1 \times 2$ выложить квадрат так, чтобы при этом не было ни одного прямого шва, соeдиняющего противоположные стороны квадрата и идущего по краям плиток?
\end{task}

\begin{solution}

\end{solution}

\begin{task}
Дана арифметическая прогрессия из $22$ различных натуральных чисел, каждое из которых является точной степенью (то есть степенью натурального числа, большей $1$). Докажите, что разность этой прогрессии больше $2010$.
\end{task}

\begin{task}
В строку выписано $1999$ натуральных чисел. Во вторую строку под каждыми двумя соседними числами выписали их наибольший общий делитель. Аналогичным образом получили третью, четвёртую и так далее строки. Может ли $1000$-я строка состоять из $1000$ последовательных чисел в некотором порядке? 
\end{task}

\begin{task}
Дано $2n + 1$ чисел ($n$ - натуральное), среди которых одно число равно $0$, два числа равны $1$, два числа равны $2$, $\dots$, два числа равны $n$. Для каких $n$ эти числа можно записать в одну строку так, чтобы для каждого натурального $m$ от $1$ до $n$ между двумя числами, равными $m$, было расположено ровно  $m$ других чисел?
\end{task}

\begin{task}
Если в таблице размером $2n \times 2n$ клеток стоят $3n$ звёздочек, то можно вычеркнуть $n$ строк и $n$ столбцов так, что все звёздочки будут вычеркнуты. Докажите это.
\end{task}

\end{document} 

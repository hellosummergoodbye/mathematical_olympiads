\documentclass[12pt]{article} 

\usepackage{geometry}
\geometry{
	paper=a4paper, 
	top=2cm, 
	bottom=2cm, 
	left=2cm, 
	right=2cm, 
	headheight=14pt, 
	footskip=1.4cm, 
	headsep=10pt, 
	%showframe, 
}

\usepackage{lmodern}
\usepackage{anyfontsize}
\usepackage{t1enc}
\usepackage{mathptmx}
\usepackage[T1,T2A]{fontenc}
\usepackage[utf8]{inputenc}
\usepackage[russian,english]{babel}
\usepackage{amsmath,amsfonts,amssymb,amsthm} 
\usepackage{mathtools}
\usepackage{fontspec}
\setmainfont{cmun}[
Extension=.otf,
UprightFont=*rm,
ItalicFont=*ti,
BoldFont=*bx,
BoldItalicFont=*bi] 

\begin{document}
\fontsize{12}{12}\selectfont

\title{\bf \huge Часть VI}
\date{Update on \today}
\maketitle 

\begin{task}
В центре каждой клетки шахматной доски стоит по фишке. Фишки переставили так, что попарные расстояния между ними не уменьшились. Докажите, что в действительности попарные расстояния не изменились. 
\end{task}

\begin{task}
В каждой клетке таблицы $10 \times 10$ записано число. В каждой строке подчеркнули наибольшее число (или одно из наибольших, если их несколько), а в каждом столбце - наименьшее (или одно из наименьших). Оказалось, что все подчёркнутые числа подчёркнуты ровно два раза. Докажите, что все числа, записанные в таблице, между собой равны.
\end{task}

\begin{task}
По кругу расставлены $10$ железных гирек. Между каждыми соседними гирьками находится бронзовый шарик. Масса каждого шарика равна разности масс соседних с ним гирек. Докажите, что шарики можно разложить на две чаши весов так, чтобы весы уравновесились. 
\end{task}

\begin{task}
Двое играют на доске $19 \times 94$ клеток. Каждый по очереди отмечает квадрат по линиям сетки (любого возможного размера) и закрашивает его. Выигрывает тот, кто закрасит последнюю клетку. Дважды закрашивать клетки нельзя. Кто выиграет при правильной игре и как надо играть?
\end{task}

\begin{task}
В турнире участвуют $2m$ команд. В первом туре встретились некоторые $m$ пар команд, во втором - другие $m$ пар. Докажите, что после этого можно выбрать $m$ команд, никакие две из которых ещё не играли между собой.
\end{task}

\begin{task}
В каждой вершине выпуклого многогранника сходятся три грани. Каждая грань покрашена в красный, жёлтый или синий цвет. Докажите, что число вершин, в которых сходятся грани трёх разных цветов, чётно.
\end{task}

\begin{task}
На доске написаны несколько чисел. Известно, что квадрат любого записанного числа больше произведения любых двух других записанных чисел. Какое наибольшее количество чисел может быть на доске?
\end{task}

\begin{task}
Круг разделён радиусами на $2n$ конгруэнтных секторов, $n$ из которых синие, а остальные - красные. В синие сектора, начиная с некоторого, по ходу часовой стрелки последовательно вписаны натуральные числа от $1$ до $n$. В красные сектора, начиная с некоторого, против хода часовой стрелки тоже последовательно вписаны числа от $1$ до $n$. Докажите существование полукруга, в сектора которого вписаны все числа от $1$ до $n$.
\end{task}

\begin{solution}

\end{solution}

\end{document} 

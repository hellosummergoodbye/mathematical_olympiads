\documentclass[12pt]{book} 

\usepackage{geometry}
\geometry{
	paper=a4paper, 
	top=2cm, 
	bottom=2cm, 
	left=2cm, 
	right=2cm, 
	headheight=14pt, 
	footskip=1.4cm, 
	headsep=10pt, 
	%showframe, 
}

\usepackage{lmodern}
\usepackage{anyfontsize}
\usepackage{t1enc}
\usepackage{mathptmx}
\usepackage[T1,T2A]{fontenc}
\usepackage[utf8]{inputenc}
\usepackage[russian,english]{babel}
\usepackage{amsmath,amsfonts,amssymb,amsthm} 
\usepackage{mathtools}
\usepackage{fontspec}
\setmainfont{cmun}[
Extension=.otf,
UprightFont=*rm,
ItalicFont=*ti,
BoldFont=*bx,
BoldItalicFont=*bi] 

\begin{document}
\fontsize{12}{12}\selectfont

\title{\bf \huge Часть VIII}
\date{Update on \today}
\maketitle 


\begin{task}
На совместной конференции партий лжецов и правдолюбов в президиум было избрано $32$ человека, которых рассадили в четыре ряда по $8$ человек. В перерыве каждый член президиума заявил, что среди его соседей есть представители обеих партий. Известно, что лжецы всегда лгут, а правдолюбы всегда говорят правду. При каком наименьшем числе лжецов в президиуме возможна описанная ситуация? (Два члена президиума являются соседями, если один из них сидит слева, справа, спереди или сзади от другого).
\end{task}

\begin{task}
Верно ли, что из произвольного треугольника можно вырезать три равные фигуры, площадь каждой из которых больше четверти площади треугольника? (почти решено)
\end{task}

\begin{task}
Дана таблица $n \times n$, столбцы которой пронумерованы числами от $1$ до $n$. В клетки таблицы расставляются числа $1, \dots, n$ так, что в каждой строке и в каждом столбце все числа различны. Назовем клетку хорошей, если число в ней больше номера столбца, в котором она находится. При каких $n$ существует расстановка, в которой во всех строках одинаковое количество хороших клеток?
\end{task}

\begin{task}
Существуют ли такие $14$ натуральных чисел, что при увеличении каждого из них на $1$ произведение всех чисел увеличится ровно в $2008$ раз?
\end{task}

\begin{task}
Дана доска $15 \times 15$. Некоторые пары центров соседних по стороне клеток соединили отрезками так, что получилась замкнутая несамопересекающаяся ломаная, симметричная относительно одной из диагоналей доски. Докажите, что длина ломаной не больше $200$.
\end{task}

\begin{task}
Докажите, что нельзя занумеровать рёбра куба числами $1, 2, \dots, 11, 12$ так, чтобы для каждой вершины сумма номеров трёх выходящих из неё рёбер была одной и той же. 
\end{task}

\begin{task}
Квадрат разрезан на равные прямоугольные треугольники с катетами $3$ и $4$ каждый. Докажите, что число треугольников чётно. 
\end{task}

\begin{task}
На полу комнаты площадью $24$ расположены три ковра (произвольной формы) площади $12$ каждый. Тогда площадь пересечения некоторых двух ковров не меньше $4$. 
\end{task}

\begin{task}
Дано $21$ девятиэлементное подмножество $30$-элементного множества. Тогда какой-то элемент $30$-элементного множества содержится по крайней мере в семи данных подмножествах.
\end{task}

\begin{task}
На острове Невезения с населением $96$ человек правительство решило провести пять реформ. Каждой реформой недовольна ровно половина всех граждан. Гражданин выходит на митинг, если он недоволен более чем половиной всех реформ. Какое максимальное число людей правительство может ожидать на митинге? 
\end{task}

\begin{task}
Два муравья проползли каждый по своему замкнутому маршруту на доске $7 \times 7$. Каждый полз только по сторонам клеток доски и побывал в каждой из $64$ вершин клеток ровно один раз. Каково наименьшее возможное число таких сторон, по которым проползали и первый, и второй муравьи?
\end{task}

\begin{task}
Среди $1977$ монет $50$ фальшивых. Каждая фальшивая монета отличается от настоящей на один грамм (в ту или в другую сторону). Имеются чашечные весы со стрелкой, показывающей разность масс грузов на чашках. За одно взвешивание про одну выбранную монету нужно узнать, фальшивая она или настоящая. Научитесь это делать!
\end{task}

\begin{task}
В клетках квадрата $9 \times 9$ стоят неотрицательные числа. Сумма чисел в любых двух соседних строках не меньше $20$, а сумма чисел в любых двух соседних столбцах не превосходит $16$. Чему может быть равна сумма чисел во всей таблице?
\end{task}

\begin{task}
Расстоянием между двумя клетками бесконечной шахматной доски назовем минимальное число ходов в пути короля между этими клетками. На доске отмечены три клетки, попарные расстояния между которыми равны $100$. Сколько существует клеток, расстояния от которых до всех трех отмеченных равны $50$?
\end{task}

\end{document} 

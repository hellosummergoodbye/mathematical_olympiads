\documentclass[fleqn,oneside]{book} 

\usepackage{geometry}
\geometry{
	paper=a4paper, 
	top=2cm, 
	bottom=2cm, 
	left=2cm, 
	right=2cm, 
	headheight=14pt, 
	footskip=1.4cm, 
	headsep=10pt, 
	%showframe, 
}

\usepackage{lmodern}
\usepackage{anyfontsize}
\usepackage{t1enc}
\usepackage{mathptmx}
\usepackage[T1,T2A]{fontenc}
\usepackage[utf8]{inputenc}
\usepackage[russian,english]{babel}
\usepackage{amsmath,amsfonts,amssymb,amsthm} 
\usepackage{mathtools}
\usepackage{fontspec}
\setmainfont{cmun}[
Extension=.otf,
UprightFont=*rm,
ItalicFont=*ti,
BoldFont=*bx,
BoldItalicFont=*bi] 

\begin{document}
\fontsize{12}{12}\selectfont

\title{\bf \huge Часть III}
\date{Update on \today}
\maketitle 

\chapter*{Принцип Дирихле}

\begin{enumerate}
\item На доске написаны $49$ натуральных чисел. Все их попарные суммы различны. Докажите, что наибольшее из чисел больше $600$.
\item Из целых чисел от $0$ до $1000$ выбрали $101$ число. Докажите, что среди модулей их попарных разностей есть десять различных чисел, не превосходящих $100$.
\end{enumerate}



\chapter*{Отсоритровать}

\begin{enumerate}
\item Прямоугольник разбили на $121$ прямоугольную клетку десятью вертикальными и десятью горизонтальными прямыми. У $111$ клеток периметры целые. Докажите, что и у остальных десяти клеток периметры целые.
\item В каждой клетке доски $8 \times 8$ написали по одному натуральному числу. Оказалось, что
при любом разрезании доски на доминошки суммы чисел во всех доминошках будут
разные. Может ли оказаться, что наибольшее записанное на доске число не больше $32$?
(Доминошкой называется прямоугольник, состоящий из двух клеток.)
\item В квадрате $7 \times 7$ клеток размещено $16$ плиток размером $1 \times 3$ и одна плитка $1 \times 1$. Докажите, что плитка $1 \times 1$ либо лежит в центре, либо примыкает к границам квадрата. 
\item В некоторых клетках квадрата $11 \times 11$ стоят плюсы, причём всего плюсов чётное количество. В каждом квадратике $2 \times 2$ тоже чётное число плюсов. Докажите, что чётно и число плюсов в $11$ клетках главной диагонали квадрата.
\item Назовем натуральное число хорошим, если среди его делителей есть ровно два простых числа.
Могут ли $18$ подряд идущих натуральных чисел быть хорошими?
\end{enumerate}

\end{document} 

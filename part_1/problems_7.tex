\documentclass[12pt]{article} 

\usepackage{geometry}
\geometry{
	paper=a4paper, 
	top=20mm, 
	bottom=20mm,  
	right=25mm, 
	nohead, 
	nofoot,  
	%showframe, 
}

%\usepackage{lmodern}
%\usepackage{anyfontsize}
%\usepackage{t1enc}
%\usepackage{mathptmx}
\usepackage[T1,T2A]{fontenc}
\usepackage[utf8]{inputenc}
\usepackage[unicode, pdftex]{hyperref}
\usepackage[russian, english]{babel}
\usepackage{amsmath,amsfonts,amssymb,amsthm} 
\usepackage{setspace}
\usepackage{mathtools}
	
%\usepackage{fontspec}
%\setmainfont{cmun}[
%Extension=.otf,
%UprightFont=*rm,
%ItalicFont=*ti,
%BoldFont=*bx,
%BoldItalicFont=*bi]


\newtheorem*{task}{Задача}

\newenvironment{solution}{\paragraph{Решение:}}{\hfill$\square$} 

\begin{document}
\fontsize{12}{12}\selectfont

\title{\bf \huge Часть VII}
\date{Update on \today}
\maketitle 

\begin{task}
Можно ли в таблице $11 \times 11$ расставить натуральные числа от $1$ до $121$ так, чтобы числа, отличающиеся друг от друга на единицу, располагались в клетках с общей стороной, а все точные квадраты попали в один столбец? 
\end{task}

\begin{task}
Для каких $N$ можно расставить в клетках квадрата $N \times N$ действительные числа так, чтобы среди всевозможных сумм чисел на парах соседних по стороне клеток встречались все целые числа от $1$ до $2(N-1)N$ включительно (ровно по одному разу)?
\end{task}

\begin{task}
Можно ли расставить по окружности числа $1, 2, 3, \dots, 100$ так, чтобы любые два соседних числа различались не более чем на $2$?
\end{task}

\begin{task}
В вершинах $33$-угольника записали в некотором порядке целые числа от $1$ до $33$. Затем на каждой стороне написали сумму чисел в ее концах. Могут ли на сторонах оказаться $33$ последовательных целых числа (в каком-нибудь порядке)? Тот же вопрос для $32$-угольника.
\end{task}

\begin{task}
На плоскости расположен квадрат и невидимыми чернилами нанесена точка $P$. Человек в специальных очках видит точку. Если провести прямую, то он отвечает на вопрос, по какую сторону от неё лежит $P$ (если $P$ лежит на прямой, то он говорит, что $P$ лежит на прямой). Какое наименьшее число таких вопросов необходимо задать, чтобы узнать, лежит ли точка $P$ внутри квадрата? 
\end{task}

\begin{task}
Даны $1000$ дробей $1, \frac{1}{2}, \frac{1}{3}, \dots, \frac{1}{1000}$.
\begin{enumerate}
\item Можно ли из них выбрать $8$ дробей, которые образуют арифметическую прогрессию?
\item Можно ли это сделать так, чтобы знаменатель наибольшей выбранной дроби отличался от знаменателя наименьшей выбранной дроби меньше чем на $600$?
\end{enumerate}
\end{task}

\begin{task}
В волейбольном турнире с участием $73$ команд каждая команда сыграла с каждой по одному разу. В конце турнира все команды разделили на две непустые группы так, что любая команда первой группы одержала ровно $n$ побед, а любая команда второй группы - ровно $m$ побед. Могло ли оказаться, что $m \neq n$?
\end{task}

\end{document}
